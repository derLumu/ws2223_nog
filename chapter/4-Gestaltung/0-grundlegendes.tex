\chapter{Gestaltung SOLL-Zustand}\label{sec:gestaltung}

Nach der Erfassung des IST-Zustandes erfolgt nach dem Prozess der NOG nun die Definition eines SOLL-Zustandes.
Dieser wird auf den ermittelten Informationen über den IST-Zustand, sowie den Aspekten und Kriterien nutzungsorientierter Gestaltung aufbauen.

Zunächst wird hierzu das bereits erstellte IST-Szenario in ein SOLL-Szenario umgewandelt.
Dabei bleiben Aufgabe und Kontext erhalten, die Handelnden verfügen nun jedoch über optimierte technische Hilfe.
Aus diesem SOLL-Szenario wird sodann ein Prototyp erstellt und vorgestellt.

In der Realität verfüge dieser Abschnitt über mehrere SOLL-Szenarien und einen umfangreicheren Prototyp.
Auf Grund der Kompaktheit der Veranstaltung und dem Umfang dieses Dokuments wird jedoch auf zusätzliche Szenarien verzichtet.
Ebenso werden nur die für das gegebene Szenario wichtigen Ausschnitte des Prototyps gezeigt. 