\section{Kontext}

Dieser Abschnitt definiert den in diesem Dokument bearbeiteten Nutzungskontext, den (Digitalen) Meldekopf des Finanzamts Kassel.
nach Erläuterung der Motivation folgt eine Übersicht über das Finanzamt Kassel, sowie die Tätigkeiten eines Meldekopfes.

\subsection{Motivation}

Anfang November 2022 sollten Kontexte für eine exemplarische Durchführung des Prozesses der NOG gefunden werden.
Der Autor ist als dualer Student beim Finanzamt Kassel beschäftigt und bearbeitet zur Zeit ein Semesterprojekt am Fachgebiet Software Engineering der Universität Kassel in Kooperation mit dem Finanzamt Kassel. 
Dabei handelt es sich um die Digitalisierung der Einsatzzentrale von Großdurchsuchungen der Steuerfahndung Kassel.
Zur Zeit der Kontextwahl befand sich dieses Projekt in der Phase der Anforderungsfindung.
Somit bot es sich an, den Inhalt des Semesterprojekts auch auf die Veranstaltung "Nutzungsorientierte Gestaltung" zu übertragen.

Da das Finanzamt Kassel sensibel Daten verarbeitet, sind dies betreffende Inhalte im Folgenden teilweise aus Gründen der Diskretion gekürzt, abgeändert oder geschwärzt.

\subsection{Meldekopf}

Das Finanzamt Kassel ist das nördlichste der 47 hessischen Finanzämter.
Unter anderem in Kassel angesiedelt ist die Steuerfahndung für Nordhessen.
Einige Male im Jahr führt diese größere Durchsuchungsmaßnahmen durch, für welche eine zentrale Organisation, eine Einsatzzentrale, von Nöten ist.
Diese Einsatzzentrale oder auch Meldekopf besteht aus mehreren Personen, welche mit Hilfe von Papier Unterlagen und Telefonaten die Durchsuchungen der verschiedenen Objekte organisieren.

Das Vorbereiten und Organisieren von Großdurchsuchungen ist der Kontext der hier vorgestellten Durchführung des nutzungsorientierten Prozesses.
Dabei soll die Notwendigkeit von Papier Unterlagen auf ein Minimum reduziert werden, sowie die Übersichtlichkeit der aktuellen Situation verbessert werden.
Des weiteren soll ein integriertes Protokolliersytem eingeführt werden, welche des der Behörde erlaubt, den Ablauf einer Durchsuchungen im Nachhinein genau nachzuvollziehen. 
Auf weitere Ideen und Konzepte des Produkts, welche im Zusammenhang mit NOG stehen wird im weiteren Verlauf dieser Arbeit eingegangen.