\chapter{Erfassung IST-Zustand}\label{sec:anwendung}

Dieses Kapitel umfasst die Erfassung des IST-Zustandes für den Kontext "Digitaler Meldekopf".
Auf Grund des zeitlichen Rahmens des Semesters ist nur eine Iteration des nutzungsorientierten Prozesses möglich.
Deshalb wird der Prozess der nutzungsorientierten Gestaltung hier und in den folgenden Kapiteln sequentiell bearbeitet.

Zunächst folgt ein Abschnitt über das mit Frau Christensen geführte Interview im Kontext.
Dieser besteht aus einer Dokumentation dieses Interviews und den daraus gezogenen Schlüssen.
Aufbauend auf dem Interview im Kontext und den im Zuge dessen gewonnen Informationen über die Personen im Kontext wird daraufhin eine für diese stellvertretende Persona vorgestellt.
Des Weiteren folgt die Erläuterung eines Artefaktmodelles, einem Ausschnitt aus dem Protokoll einer Großdurchsuchung.

Darauf aufbauend wird letzten Endes ein IST-Szenario verfasst, in welchem die erwähnte Persona einer gängigen Tätigkeit des Kontextes nachgeht.
