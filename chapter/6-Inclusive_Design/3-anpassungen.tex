\section{Anpassungen}

In diesem Abschnitt wird die Software mit Hilfe der Inklusive Design Support Cards \cite{ITToolkit} auf die Probe gestellt.
Dabei wird überlegt, welche Fälle bereits in der Software abgedeckt sind.
Ebenso wird für nicht abgedeckte Fälle überlegt, ob und wie diese von der Software abgedeckt werden können.

\subsubsection{Physical Context}

Der physikalische Kontext bezieht sich auf den Ort, an dem die Software eingesetzt wird.
Da es sich bei der Software um eine Arbeitsanwendung handelt, besitzt diese einen eingeschränkten Rahmen an Einsatzorten.

Primär wird sie mit den idealen Bildschirmgrößen in den Räumen des Finanzamts eingesetzt.
Soll jedoch die Persona Lina mit an einem Meldekopf teilhaben, so muss die Software auch auf einem einzigen Bildschirm im Home-Office funktionsfähig sein.
Dies ist gegeben.
Lina kann Eingabemaske und Übersichtsbildschirm in zwei separaten Tabs öffnen. 
Je nachdem welchen Tab sie benötigt, kann der andere Tab in den Hintergrund gestellt werden.

Als weiteren Einsatzort der Software ist in Zukunft der mobile Einsatz in Objekten oder im Auto zu nennen.
Perspektivisch soll der Digitale Meldekopf um eine Smartphone Version ergänzt werden.
Über diese können an der Durchsuchung beteiligte Personen über ihr Smartphone die aktuelle Timeline der Durchsuchung einsehen.
Das Eingeben von Informationen wird am Smartphone jedoch nicht möglich sein.
Hierzu ist zum aktuellen Zeitpunkt jedoch ebenso kein Anwendungsfall bekannt.

\subsubsection{Social Contex}

Die Problemstellung des sozialen Kontextes kann der digitale Meldekopf ebenfalls lösen.
Auch hier kommt ihm die Nutzung als Arbeitssoftware zu gute.
Die Benutzung mit Freunden, der Familie oder in einer Menschenmenge ist somit ausgeschlossen.
Des Weiteren kann die Software allein, oder im Team mit Kollegen genutzt werden.
Die alleinige Nutzung ist jedoch auf Grund der Menge an Arbeit im Zuge von Großdurchsuchungen nicht empfohlen.

\subsubsection{Temporary/Situational Limit}

Einschränkungen aller Art sind bei der Entwicklung von Software ein großes Thema.
Es muss entschieden werden, welche Einschränkungen von der Software unterstützt werden.
Nicht alle Einschränkungen werden vom Digitalen Meldekopf unterstützt.

Menschen ohne Sehvermögen können den Digitalen Meldekopf nur eingeschränkt bedienen.
Hierzu kann die von allen Betriebssystemen unterstützte Text-To-Speech Technologie verwendet werden.
Der geschriebene Inhalt wird den Nutzenden somit vorgelesen.
Eine Hilfe zu dieser Einschränkung enthält der Digitale Meldekopf jedoch nicht.
Durch den Einsatz der Text-To-Speech Technologie wird die Effizient des Digitalen Meldekopfs stark geschwächt.
Mit Hilfe geschickter Arbeitsteilung kann eine Erblindete Person jedoch ohne sinkende Effizienz in den Digitalen Meldekopf integriert werden.
Sie übernimmt den Telefonverkehr, während ein(e) Kolleg*in die Bedienung der Eingabemaske übernimmt.

Ist eine Stumme Person an Digitalen Meldekopf beteiligt, so kann ähnlich vorgegangen werden. 
Für das Nutzen der Software ist Sprache nicht notwendig.
So kann eine Stumme Person die Bedienung der Software übernehmen, während ein(e) Kolleg*in den Telefonverkehr übernimmt.
Selbes Vorgehen ist ebenfalls möglich, sollte eine Taube Person am Meldekopf teilhaben.

Sollte die Nutzung von Armen oder Händen einer Person eingeschränkt sein, so kann der Digitale Meldekopf nicht uneingeschränkt genutzt werden.
Das Eingeben von Informationen wird in allen Fällen durch eine sinkende Eingabegeschwindigkeit beeinträchtigt.
Über den Einsatz einer Speech-To-Text Technologie wurde nachgedacht.
Diese Technologien beziehen ihre Funktionalitäten jedoch von externen Servern.
Dies ist auf Grund der Vertraulichkeit der verarbeiteten Daten nicht erlaubt.
Eine Speech-To-Text Funktion kann also erst dann angeboten werden, wenn innerhalb der hessischen Finanzverwaltung ein solches Speech-To-Text Modell erstellt wird.

\subsubsection{Role of Technology}

Die Rollen des Digitalen Meldekopf sind das Sammeln und Zusammenfassen von Informationen, sowie ein Transport von Informationen zwischen Objekten und Meldekopf.
Diese Rollen erfüllt die vorgestellte Softwarelösung in allen Punkten.
An keiner Stelle findet die Bereitstellung unnützer Informationen statt.

\subsubsection{Conditions}

Des Weiteren Soll die Software bei unterschiedlichen Bedingungen genutzt werden.
Wetter und Temperatur tragen keinen Einfluss auf den Nutzen der Software.
Für Einsatzzeiten, zu denen des dunkel ist, ist zukünftig ein dunkler Modus geplant.
Dieser versucht die Oberfläche auch bei Dunkelheit für die Augen verträglich darzustellen.

Hier ebenso zu nennen sind Hintergrundgeräusche.
Diese treten beispielsweise bei der Persona Lina auf.
Bei anfallenden Hintergrundgeräuschen kann die Nutzung der Software durch Ablenkungen beeinträchtigt werden.
Dies kann jedoch durch das Verwenden von Kopfhörern vermieden werden.