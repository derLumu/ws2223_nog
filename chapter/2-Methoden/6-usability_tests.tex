\section{Usability Tests}

Um das entwickelte Softwarekonzept zu testen, werden Usability Tests mit den Kontextpersonen durchgeführt.
Usability Tests lassen Kontextpersonen die Prototypen ohne direkte Hilfe der Entwickelnden testen.
Diese Tests folgen einem festen Ablauf, welcher im Folgenden erläutert wird.

\subsection{Vorbereitung}

Zunächst müssen einige Voraussetzungen für Usability Tests erfüllt werden.
Es benötigt einen ruhigen Ort, Testpersonen, sowie je einer Person zur Moderation und Protokollführung.
Die beiden letzteren müssen die Software kennen, die Testpersonen dürfen dies nicht.

Des weiteren müssen Aufgabenstellungen vorbereitet werden, welche sich nah an Beobachteten Aufgaben orientieren und einen möglichst großen Teil der Anwendung abdecken.
Diese sollen in angemessener Zeit lösbar sein und eine natürliche Abfolge an Aktionen beinhalten.
Ebenso soll ein Szenario als Rahmen des Tests vorbereitet werden.
Dies soll den Nutzenden Informationen über die Umstände geben, in denen sie die Aufgabenstellungen abarbeiten sollen.

Ebenso zu empfehlen ist ein vorausgehender Pilottest, welcher sicherstellt, dass die Aufgaben verständlich sind.
Dieser kann ebenfalls feststellen, ob große technische Probleme in der Software vorliegen oder alle Materialien zum Lösen der Aufgaben vorhanden sind.

\subsection{Durchführung}

Zu Beginn des Tests soll der Testperson klar gemacht werden, dass die Software und nicht die Benutzenden geprüft werden.
Daraufhin werden die Testpersonen gebeten, stehts geplante Aktionen und erwartete Rückmeldungen vom System zu kommunizieren.
Im Falle einer Benutzung durch mehrere Personen muss jede Aktion begründet werden.

Daraufhin werden den Testpersonen die Aufgaben und der erstellte Kontext erklärt.
Sie können daraufhin anfangen die Aufgaben zu bearbeiten.
Alle Tätigkeiten werden von der Protokoll führenden Person notiert, um diese zu einem späteren Zeitpunkt auszuwerten.
Der Moderator greift während der Bearbeitung nur im Notfall ein.
Dies könnte beispielsweise durch ein absolutes nicht Weiterkommen der Testpersonen gegeben sein.

\subsection{Nachbereitung}

Schlussendlich wird eine Nachbereitung durchgeführt.
Hier werden die Testpersonen zunächst nach ihrer grundlegenden Einschätzung des Programms gefragt.
Über diese hinaus werden Probleme, Überraschungen und Mängel besprochen und evaluiert.
Diese Diskussion wird final in einen Bericht zusammengefasst, welcher Verbesserungsvorschläge in einer konkreten, strukturierten und konstruktiven Art und Weise enthält.