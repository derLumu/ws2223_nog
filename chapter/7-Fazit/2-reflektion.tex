\section{Reflektion}

Der gewählte Kontext überschreitet den für diese Arbeit benötigten Umfang um ein weites.
Diese Erkenntnis wurde beim Zusammentragen der über das Semester gesammelten Informationen immer deutlicher.
Trotzdem wurde der nutzungsorientierte Prozess am Beispiel des Digitalen Meldekopf mit einer hohen Sorgfalt umgesetzt.

Jeder der getätigten Schritte forderte den Autor dabei auf eigene Weise.
Das Interview Im Kontext konnte zwar nicht direkt nach dem originalen Konzept ablaufen.
Trotzdem wurden die benötigten Informationen aus diesem gewonnen.
Die Gesamtzeit der Beobachtung im Laufe des Semesters betrug über 15 Stunden, aufgeteilt auf drei Großdurchsuchungen.
Dabei konnte Anfangs nur Auffälliges notiert werden.
In den späteren Durchsuchungen konnten bereits einige Konzepte und Ideen praktisch getestet werden.
Dadurch entstand eine Art ungewollte Iteration in der Entwicklung, welche jedoch nur den Entwickler selbst betraf.

Im Zuge der Rückmeldungen der Kontextperson wurde klar, dass trotz enormen in dieses Projekt geflossenem Aufwand nicht alle Ideen und Konzepte zielführend waren.
jedoch war es durch einen ausgezeichneten Austausch zwischen dem Entwickler und den Kontextpersonen möglich, jegliche Probleme und Wünsche in umsetzbare Konzepte umzuwandeln.
Es ist von Konzepten die Rede, da wie in \autoref{sec:erweiterung} zu erkennen ist, zum aktuellen Zeitpunkt nicht all diese Konzepte implementiert wurden.

Schlussendlich ist jedoch ein stark positiver Ausgang des nutzungsorientierten Prozesses festzuhalten.
Die durch ihn gewonnenen Informationen und Methoden haben großen Einfluss auch auf die Ausgestaltung der parallelen Projektarbeit.
Somit konnte der Autor nicht nur den durch NOG verursachten Mehraufwand spüren, sondern ebenfalls die aus diesem Aufwand resultierenden Früchte ernten.
Somit hat der nutzungsorientierte Prozess direkt eine sehr positive Auswirkung auf das Endprodukt.