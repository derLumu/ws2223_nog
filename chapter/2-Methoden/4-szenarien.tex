\section{Szenarien}\label{sec:sezenarien}

Szenarien werden verwendet, um konkrete Situationen textuell zu beschreiben.
Dabei können real Beobachtete Situation beschrieben werden.
Es können jedoch auch ideale Situationen in Form von SOLL-Szenarien erstellt werden.

IST- und SOLL-Szenarien folgen einem gemeinsamen Leitfaden.
Sie verfügen grundlegend über einen Titel, einen Aktor und eine Aufgabenstellung.
Das Szenario beschreibt die Ausführung der Aufgabenstellung durch den Aktor.
Dabei werden Nutzungskontext, Kommunikation, Hierarchien und Herausforderungen in das Szenario eingearbeitet.
Szenarien sollen Abläufe nicht bewerten.
Ihr Ziel ist es eine möglichst neutrale Beschreibung einer Situation darzustellen.

Als Aktor der Szenarien werden in vielen Fällen bereits erstellte Personas verwendet.
Dies bietet sich an, da so sichergestellt wird, dass Handlungen innerhalb des Szenarios nicht übermäßig durch die handelnde Person beeinflusst werden, da Personas einen durchschnittlichen Nutzenden darstellen.
Szenarien müssen nicht im Fließtext vorliegen.
Es ist ebenso möglich die benannten Anforderungen Stichpunktartig oder auf andere Weise zu erfüllen.

\subsubsection{IST-Szenario}

Ein IST-Szenario spiegelt den momentanen Zustand eines Kontextes wider.
Es wird aus den im Interview im Kontext und aus Modellen gewonnenen Informationen erstellt.
Diese IST-Szenarios stellen die Grundlage für zu entwickelnde Software dar.
Im Normalfall werden im Zuge des nutzungsorientierten Prozesses mehrere IST-Szenarios erstellt.

\subsubsection{SOLL-Szenario}

Ein SOLL-Szenario gibt den idealen Ablauf einer Tätigkeit wider.
Jedes SOLL-Szenario baut im idealen Fall auf einem IST-Szenario auf, welches über optimierbare Techniknutzung verfügt.
Diese optimierte Techniknutzung steht den Nutzenden im SOLL-Szenario zur Verfügung.
Dabei muss darauf geachtet werden, dass die beschriebene neue Technik den Aspekten der NOG entspricht.