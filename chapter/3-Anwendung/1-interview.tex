\section{Interview im Kontext}\label{sec:interview-praktisch}

Dieser Abschnitt enthält alle Informationen zum Interview im Kontext mit Frau Christensen. 
Zunächst folgt eine Dokumentation des Interviews, welche anhand von Mitschriften und Erinnerungen an dieses Gespräch entsteht.
Darauf folgt eine Auswertung des Interviews, in der Punkte herausgestellt werden, welche bei der Entwicklung eines SOLL-Zustandes berücksichtigt werden müssen.

Der Ablauf des Interviews musste in diesem Kontext leicht modifiziert werden.
Das Interview fand im Laufe einer realen Großdurchsuchung statt.
Im Zuge dieser ist stehts ein hohes Maß an Hektik präsent.
Um diese nicht weiter zu steigern, wurden Zwischenfragen zu Arbeitsschritten ans Ende der Durchsuchung verschoben.
Somit wurde zunächst über sechs Stunden eine reine Beobachtung durchgeführt.

Auf Grund der abgewandelten Durchführung des Interviews im Kontext enthält dieser Abschnitt nicht die vollständige Dokumentation der gesamten Interviewzeit.
Es werden im Folgenden also einige im Zuge der Durchsuchung beobachtete Fakten und deren Bedeutung aufgelistet.

Des Weiteren handelt es sich bei den Beobachtungen um hoch sensible Daten.
Jegliche Namen von Personen, Betrieben und weiterem müssen und sind somit geändert.
Um den Lesenden eine lebhafte Atmosphäre zu schaffen, werden Namen aus der Zeichentrick Serie \textit{SpongeBob Schwammkopf} verwendet.

\subsection{Beschreibung der physischen Umgebung}

Gegenstand der Durchsuchung ist der Verdacht auf Steuerhinterziehung in der Belegschaft der Krossen Krabbe.
Hierzu werden die Krosse Krabbe selbst, sowie die Wohnungen von Eugene Krabs, Robert Schwammkopf und Thaddäus Tentakel durchsucht.

Es ist der 22.09.2029 um 12 Uhr mittags.
Der Meldekopf sitzt in einem Besprechungsraum für ca. 20 Personen des Finanzamts Kassel.
Er besteht am heutigen Tag aus zwei Personen, Frau Vogel und Herrn Müller-Hall.
Der große Konferenztisch ist voll mit Aktenordnern, einem Notebook, zwei Kaffeetassen und einer Tüte Kräppel.
An einer Wand hängen DIN-A4-Blätter, bedruckt mit den wichtigsten Informationen zu allen Objekten.
Die Namen der an der Durchsuchung beteiligten Personen kleben in Form von Post-Its auf den jeweiligen Blättern.
Das Notebook ist mit einem SmartBoard verbunden, auf welchem eine Excel-Datei zwecks Protokollierung der Geschehnisse geöffnet ist.
Frau Vogel und Herrn Müller-Hall haben jeweils ihre Handys griffbereit, um von den durchsuchenden Personen erreicht werden zu können.

\subsection{Beobachtungen}

\textbf{Einsatzpläne sind nicht einheitlich:} Zu jeder Durchsuchung existieren Einsatzpläne.
Diese werden jedoch je nach zuständiger Person unterschiedlich erstellt.
Ein einfaches Einlesen dieser in ein Programm ist also nicht möglich.

\textbf{Das Protokoll besitzt Shortcuts:} Des Öfteren benutzt Frau Vogel beim Pflegen des Protokolls Shortcuts.
Beispielsweise gibt die "FA" ein und drückt die Leertaste.
Das Programm ersetzt "FA" sofort durch "Finanzamt Kassel".

\textbf{Übersicht welches Objekt bereits betreten wurde fehlt:} Zu Beginn der Durchsuchung ist es Ziel, alle Objekte zu betreten.
Vogel und  Müller-Hall besprechen alle paar Minuten, ob dieses Ziel erreicht ist.
An den Beobachter gerichtet wird eine technische Lösung für dieses Problem gewünscht.

\textbf{Oft wird das Objekt zu einer konkreten Fahndungsperson gesucht:} Im Protokoll soll zu jeder dort vermerken Information ein Objekt angegeben werden.
Der Meldekopf hat in den meisten Fällen jedoch nur die Namen der Anrufer. 
Mit diesen muss an der Wand das gewünschte Objekt befunden werden.
Dafür stehen Vogel und Müller-Hall des Öfteren von ihren Plätzen auf.

\textbf{Häufig fallen ToDos an:} Oft erfragen die Anrufenden Informationen beim Meldekopf.
Auf Grund der Hektik können diese in vielen Fällen nicht direkt beantwortet werden.
Teilweise werden ToDos auf Papier festgehalten, teilweise jedoch nur im Kopf.

\textbf{Es können neue Objekte entstehen:} Im Zuge der Durchsuchung wurden Hinweise zu einem weiteren zu durchsuchenden Objekt gefunden.
Für dieses neue Objekt wurde kein neues Papier erstellt.
Die dort tätigen Personen wurden ebenfalls irgendwo markiert.

\textbf{Übersicht welches Objekt abgeschlossen ist fehlt:} Das Selbe Problem wie beim Betreten von Objekten besteht auch beim Verlassen dieser.

\textbf{Es kann Personal vom Zoll auftreten:} Bei machen Objekten durchsucht der Zoll zur selben Zeit.
Diese Durchsuchung wird nicht durch den Meldekopf überwacht.
Trotzdem sind die Personen des Zolls in den Akten notiert.

\textbf{Es wird nie direkt protokolliert:} Auf Grund der Hektik werden Anrufe fast nie direkt dokumentiert.
Der längste Stau an zu protokollierenden Informationen betrug über eine halbe Stunde.
Dies geschah, da Frau Vogel in dieser Zeit ununterbrochen telefonierte und somit nicht gleichzeitig protokollieren konnte.

\textbf{Telefonnummern sind schwer zu finden:} Teilweise werden mit dem Fall in Verbindung stehenden Telefonnummern benötigt.
Diese befinden sich in den Akten, jedoch nicht an einem zentralen Ort.
Die Suche nach Telefonnummern kann also schnell erledigt sein, in manchen Fällen aber auch über zwei Minuten dauern.

\subsection{Auswertung}

Zum Zeitpunkt des Interviews im Kontext waren bereits erste Ideen und Konzepte der Technikumsetzung präsent.
Dies geschah, da im Zuge der Projektarbeit bereits parallel an einer Lösung ohne Einsatz des nutzungsorientierten Ansatzes gearbeitet wurde.
Dabei wurde jedoch in größten Teilen zunächst die Technik im Hintergrund der Anwendung bearbeitet.
Im Zuge der Auswertung sind einige Schlüsse aus den Beobachtungen gezogen wurden:

Es war geplant, Einsatzpläne einfach importieren zu können.
Auf Grund der Unterschiedlichkeit dieser wurde hier eine Bitte auf Verallgemeinerung angestoßen, damit diese mit Hilfe eines Parsers eingelesen werden können.
Bis zum heutigen Tag gibt es keine Rückmeldung zu dieser Bitte.

Des Weiteren wurde festgestellt, dass häufig eine Übersetzung zwischen Person und Objekt stattfinden muss.
Es bietet sich also an, diese digital zu lösen.
Die hierzu benötigten Daten liegen im Falle einer Digitalisierung vor und können somit ohne großen Aufwand abgerufen werden.

Ein weiterer auffälliger Punkt war der Zeitverlust, welcher durch unübersichtliche Anordnungen von Informationen zu Stande kam.
Oft musste durch den Raum gelaufen werden oder ein Aktenordner durchsucht werden.
Ziel muss es also sein, diese Zeit zu reduzieren.
Hierzu kann der große Bildschirm des SmartBoards verwendet werden, welcher momentan nur das Protokoll spiegelt.
Es weiteren können oft benötigte Informationen ebenfalls in die Software übertragen werden, um somit den Zugriff über eine mögliche Suchfunktion zu vereinfachen.

Schlussendlich ist anzumerken, dass diese Beobachtungen und Auswertungen nur einen Ausschnitt aus dem gesamten Prozess darstellen.
Diese Reduzierung der Informationen ist auf Grund des Umfangs dieses Dokuments nötig.
Trotzdem sind alle Beobachtungen und Erfahrungen in die Ergebnisse der nächsten Schritte eingeflossen.
Sollten dem Leser unerklärliche Schlussfolgerungen auffallen, so sind diese der Komprimierung der Informationen geschuldet.
Jede Entwurfsentscheidung ist auf einer Beobachtung oder anderen Quelle basierend.