\section{Feedback}\label{sec:feedbackPraktisch}

Im Folgenden sind die Rückmeldungen der Kontextperson, sowie eigene Beobachtungen zur Verbesserung des Prototyps aufgelistet.
Diese werden zunächst nur beschrieben.
Die Abschnitte \autoref{sec:anpassungen} und \autoref{sec:erweiterung} enthalten sodann Lösungen der benannten Probleme in Form von angepassten Prototypen.

In normalen Fällen obliegt es nicht den Entwicklern, Änderungen an dem Produkt anzustoßen.
Durch die Arbeit des Autors im Finanzamt Kassel gehört dieser jedoch zu Teilen selbst in den Kreis der Kontextpersonen.
Somit führte der Autor einen praktischen Test der Software selbst durch.
Dies resultiert in dem Vorteil, dass dieser zwischen fertigen und unfertigen Funktionen der Software unterscheiden kann.
Jedoch kann der Autor nicht die Selbstbeschreibungsfähigkeit und Lernfähigkeit seines eigenen Systems beurteilen.

\subsection{Kontextperson}\label{sec:feedbackPerson}

\textbf{Einsatzleitung:} Frau Christensen fiel bei genauerer Betrachtung der Prototypen auf, dass diese keine Möglichkeit bieten die den Einsatz leistende Behörde anzuzeigen.
Im klassischen Fall ist auf den Objektzetteln ein "+" für "Finanzamt hat Objektleitung" und ein "-" für "Eine andere Behörde hat die Objektleitung" abgedruckt.
Diese Angaben sind wichtig, wenn kritische Fragen bezüglich gefundenen Gegenständen entstehen.
Fragen dieser Art sind an die oberste Objektleitung und deren Meldekopf zu stellen.
Die Lösung dieses Problems ist in \autoref{sec:erweiterung} zu finden.


\textbf{Zeiterfassung Protokoll:} Ebenfalls wurde die Erfassung der Zeit im Protokoll beanstandet.
Die Tatsache, dass das Protokoll in vielen Fällen nicht direkt nach Eingang einer Information ausgefüllt wird, ist bereits aus \autoref{sec:interview-praktisch} bekannt.
Jedoch war eine genaue Erfassung der Zeit zu Gunsten einer einfacheren Bedienung als überflüssig betrachtet worden.
Frau Christensen stellte jedoch heraus, dass eine genaue Zeit für die weitere Arbeit mit dem Protokoll essentiell ist, da dieses häufig mit anderen Daten mit Zeitstempel abgeglichen wird.
Die Lösung dieses Problems ist in \autoref{sec:erweiterung} zu finden.

\textbf{Telefonnummern:} Eine weitere Anmerkung betrifft das Anzeigen von Telefonnummern auf dem Übersichtsbildschirm.
Diese wurden bei ersten Gesprächen mit anderen Kontextpersonen gewünscht, stellen sich jedoch als nutzlos heraus.
Die diesen Wunsch äußernde Kontextperson betrieb selbst vor längerer Zeit den Meldekopf.
Zu dieser Zeit verfügten die Personen nicht über automatisch gepflegte Kontaktangaben der Fahnder*innen in den Diensthandys.
Inzwischen sind diese vorhanden.
Die Lösung dieses Problems ist in \autoref{sec:anpassungen} zu finden.

\textbf{Referenzieren von Updates:} Ebenso wurde beanstandet, dass beim Verfassen eines Updates kein bestehendes Update referenziert werden kann.
Dies soll jedoch möglich sein, damit beispielsweise die Ausführung eines ToDos direkt mit der vorherigen Erstellung desselben verbunden werden kann.
Diese Funktion wurde in den bisherigen Prototypen unabsichtlich nicht beachtet.
Die Lösung dieses Problems ist in \autoref{sec:erweiterung} zu finden.

\textbf{Updates bearbeiten:} Als letzten Punkt nannte Frau Christensen das bearbeiten bereits abgeschickter Updates.
Dies sei notwendig, sollten Informationen für ein Update vergessen worden sein oder ein simpler Tippfehler vorliegen.
Eine vorherige Beachtung dieser Funktion fand nicht statt, da der Entwickelnde im Falle zusätzlicher Informationen stets ein neues Update verfasste.
Tippfehler waren im Konzept ebenfalls verkraftbar.
Die Lösung dieses Problems ist in \autoref{sec:anpassungen} zu finden.

\subsection{Beobachtungen}\label{sec:feedbackMe}

\textbf{Objekte Filtern:} In der praktischen Anwendung viel dem Autor auf, dass das erstellte System vom Filtern von Objekten in der Praxis unverständlich ist.
Das Filtersystem basiert im technischen Hintergrund rein auf Tags, dies wird den Nutzenden jedoch nicht klar kommuniziert.
Sie denken, ein auf Kategorien basierendes Filtersystem zu nutzen, jedoch wird im Hintergrund nur die Existenz eines Tags für ein Objekt überprüft.
Diese Unklarheit muss für das Endprodukt beseitigt werden.
Die Lösung dieses Problems ist in \autoref{sec:anpassungen} zu finden.

\textbf{Letzte Standorte:} Im Zuge einer Großdurchsuchung, welche sich über ein größeres innerdeutsches Gebiet erzog wurde festgestellt, dass die Angabe des letzten Standortes für verfügbare Personen nützlich ist.
War eine Person beispielsweise in einem Objekt in München zu Gange, wird so ausgeschlossen, dass diese zu einem Objekt in beispielsweise Hamburg verschoben wird.
Die bei der Entwicklung des Konzepts bekannten Großdurchsuchungen erstreckten sich stets über mit dem Auto gut erreichbare Distanzen.
Auf Nachfrage wurde jedoch bestätigt, dass häufiger große Distanzen zwischen Objekten auftreten.
Die Lösung dieses Problems ist in \autoref{sec:erweiterung} zu finden.

\textbf{Timeline Tags:} Ebenfalls viel auf, dass eine schnelle Zuordnung von Updates der Timeline zu Personen oder Objekten schwer viel.
Hierzu muss stets der gesamte Text des Updates überflogen werden.
Diese Texte sind der der Praxis länger vorher angenommen.
Dabei können Updates schnell ein Drittel der Timeline bedecken.
Im Zuge der Konzepterstellung wurde von maximal dreizeiligen, prägnanten Updates ausgegangen.
Die Lösung dieses Problems ist in \autoref{sec:anpassungen} zu finden.

\textbf{Hintergrund von Objekten:} Ein weiterer zu beanstandender Punkt in den Prototypen ist die Gestaltung des Hintergrunds von Objekten.
Dieser war zunächst je nach Standort des Objekts in verschiedenen Farben gefärbt.
In der Praxis stellte sich dies jedoch als unübersichtlich dar.
Die Lösung dieses Problems ist in \autoref{sec:anpassungen} zu finden.

\textbf{Rechtschreibkorrektur:} Im Zuge der längeren Benutzung des programmierten Prototypen wurde eine integrierte Rechtschreibkorrektur schmerzlich vermisst.
Durch Tippfehler wurde die Produktivität des Nutzenden stark gemindert, da das finden und Korrigieren von Schreibfehlern stets einige Zeit in Anspruch nahm.
Eine Rechtschreibkorrektur ist somit dringend gewünscht.
Die Lösung dieses Problems ist in \autoref{sec:erweiterung} zu finden.