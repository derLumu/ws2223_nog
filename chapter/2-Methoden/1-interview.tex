\section{Interview im Kontext}\label{sec:interview}

Das Interview im Kontext ist eine Methode, mit welcher die Entwickelnden Informationen über den aktuellen Zustand in einem Kontext bekommen können.
Entwickelt wurde es in der Mitte der 1990er Jahre von Holzblatt und Beyer erfunden \cite{InterviewIK}.
Ziel es Interview ist es, auf Basis von konkreten Arbeitssituationen der Nutzenden, zuverlässige und umfangreiche Daten über den Einsatz, die Nutzung und den Kontext von IT zu erheben.
Dabei wird das existierende System in Bezug auf Beziehungen, Abläufe, Widersprüche, Fehler und Überflüssigkeiten genauestens untersucht.
Ebenso entwickelt sich durch das Interview im Kontext ein Bild der Gruppe der Nutzenden, welches später beim Erstellen von Personas wichtig wird.

Das Interview im Kontext ist ein Beobachtungsinterview.
Informationen werden also nicht vorrangig durch Fragen, sondern durch Beobachtungen gewonnen.
Dabei befinden sich die Interviewenden in der Rolle eines Lehrlings, während die Kontextperson die Rolle eines Meisters einnimmt.
In dieser Rolle begleiten die Interviewenden die Kontextperson bei ihren Tätigkeiten im Kontext.
Eine detaillierte Ablaufbeschreibung des idealen Interviews im Kontext ist \autoref{ap:checkliste} zu entnehmen.
Diese kann jedoch auf den vorherrschenden Kontext angepasst werden.
Ein Beispiel einer Anpassung ist im weiteren Verlauf dieser Arbeit unter \autoref{sec:interview-praktisch} zu finden.
Über das Beobachten der Kontextpersonen sind Fragen der Interviewenden möglich.
Diese sollen jedoch konkret und nicht abstrakt formuliert werden, damit das Interview weiter so nah wie möglich an einer konkreten, realen Aufgabe stattfindet.
Des Weiteren sollen sich die Beobachtenden durch Bescheidenheit, Neugier und Aufmerksamkeit auszeichnen.

Das Interview im Kontext folgt den Grundprinzipien Kontext, Partnerschaft, Interpretation und Fokus.
Diese seien im Folgenden näher beleuchtet.

\subsection{Kontext}

Die Interviewenden müssen zu jeder Zeit des Interview den Kontext im Bezug halten.
Dabei sollen die gewonnenen Erfahrungen als laufende und nicht als zusammengefasste Erfahrungen analysiert werden.
Des Weiteren gilt es, stehts nah an realen Abläufen zu agieren und nicht an fiktiven, möglicherweise nicht die Realität , Abläufen zu arbeiten.
Es muss also ein Fokus auf Konkretem und nicht auf Abstraktem liegen.
Ebenso sollen die Interviewenden stehts zum Weiterarbeiten anregen und mit gezielten Fragen den Kontext der aktuellen Aufgabe erfassen.
Hier kann beispielsweise nach dem Grund für die Aufgabe oder einem ähnlichen Fall und dessen Ausgang gefragt werden.

\subsection{Partnerschaft}

Dieses Prinzip stellt die Zusammenarbeit zwischen den Interviewpartnern heraus.
Es sei wiederholt, dass das Interview im Kontext nicht mit der traditionell in einem Interview herrschenden Marktsituation übereinstimmt.
Die Kontextperson entscheidet, wann und wie lange über ein Thema geredet wird.
Dieses Privileg ist dieser vor dem Interview klar mitzuteilen.

Diese Unterordnung der Interviewenden hat einige Gründe:
Es soll verstanden werden, wie die Arbeit unterstützt werden kann.
Hierzu ist es essentiell, dass diese Arbeit zunächst ungefiltert betrachtet wird.
Dies ist nur möglich, wenn die Kontextperson in der Machtposition ist.
Ebenso werden gewisse Arbeitsaspekte unterbewusst ausgeführt.
Um diese wahrnehmen zu können, muss die Kontextperson nach eigenem Ermessen handeln können und ihre Aufgaben wie gewohnt erledigen.

Im Idealfall sollten die Interviewenden Fragen steht nach Beendung eines Arbeitsschritts stellen.
So wird auch die Kontextperson zum Reflektieren angeregt, was dazu führen kann, dass diese mehr Informationen explizit an die Interviewenden weitergeben kann.

\subsection{Interpretation}

Der Punkt der Interpretation wird nach dem Interview im Kontext wichtig.
Die Interviewenden müssen darüber im Klaren sein, dass sie subjektive und nicht objektive Beobachtungen gesammelt haben.
Diese müssen unter dem Aspekt der Subjektivität beurteilt werden.

Nach abgeschlossener Interpretation des Interviews sind die Ergebnisse mit der Kontextperson abzugleichen.
Diese erkennt mögliche Missinterpretation und kann diese klarstellen.
Wichtig ist, dass Design-Ideen als letztes Glied dieser Bildung von Verständnis gesehen werden sollen.
Es sollen also nicht im Zuge des Interviews direkt Ideen an die Kontextperson weitergegeben werden.

\subsection{Fokus}

Als letztes Prinzip ist Fokus zu nennen.
Hierbei geht es nicht um den Grad der Aufmerksamkeit, sondern deren Ziel.
Wichtig ist es, den Fokus auf Unerwartetes zu legen.
Als Außenstehender ist es schwer, Ausnahmen von regulären Arbeitsschritten zu unterscheiden.
Für diese Fälle sind Nachfragen geeignet.
Es muss darauf geachtet werden, dass das der Fokus nicht auf der Bestätigung der eigenen Auffassung liegt, sondern, dass die Interviewenden offen für Überraschungen und Widersprüche sind.
