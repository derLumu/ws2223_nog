\section{Zusammenfassung}

Die Durchführung des nutzungsorientierten Prozesses ist Zeitaufwendig.
Dies geht allein mit seiner Iterativen Definition einher.
Um diesen Aufwand abzuschwächen, wurde im Zuge dieser Arbeit auf Iterationen verzichtet und der nutzungsorientierten Prozesses in einen sequenziellen Rahmen gegossen.

Zu Beginn wurde in einem Interview im Kontext grundlegendes Wissen über den Kontext gewonnen.
Dieses Interview musste auf Grund des Kontextes methodisch angepasst werden.
Zwar wurde über fünf Stunden lang beobachtet, Fragen konnten jedoch erst am Ende gestellt werden.
Aus dem Interview im Kontext konnten Persona und IST-Szenario gewonnen werden.
Hierbei konnte die gelernte Methodik die vorgesehen angewandt werden.
Letztlich musste die Anzahl an Personas und Szenarien jedoch reduziert werden.

Im Zuge der Gestaltung des SOLL-Zustand konnten Gedanken und Fortschritte des zur selben Zeit bearbeiteten Semesterprojekt verwendet werden.
Somit konnte ein SOLL-Konzept schnell erarbeitet werden.
Das Verbessern dieses Konzepts begann mit einem Usability Test anhand zweier Prototypen.
Diese wurden nicht nur von der Kontextperson getestet.
In einem eine Großdurchsuchung begleitendem Versuch konnte der Autor seine eigene Software praktisch testen.
Da er selbst Erfahrungen in der Finanzverwaltung gesammelt hat, gehört dieser selbst in Teilen zu den potenziellen  Nutzenden.
Die in den Tests zu beanstandenden Funktionen wurden daraufhin verbessert und schlussendlich einer Überarbeitung im Rahmen des Inclusive Designs unterzogen.