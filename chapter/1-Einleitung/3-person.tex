\section{Personen und Tätigkeit}

Dieser Abschnitt stellt den Lesenden die im nutzungsorientierten Prozess beteiligte, im Kontext tätige Person, sowie ihre Tätigkeiten vor.
Dabei wird ein Fokus auf die zu optimierende Aufgabe, dem Organisieren und Durchführen von Großdurchsuchungen im Meldekopf, gelegt.
Zu beachten ist, dass aus Gründen der Diskretion der Name der erwähnten Person geändert wurde.

\subsubsection{Kontextperson}

Als Kontextperson, beziehungsweise Ansprechperson aus dem Kontext, steht Hannah Christensen zur Verfügung.
Frau Christensen arbeitet seit vielen Jahren in der hessischen Finanzverwaltung und ist zur Zeit in der Steuerfahndung Kassel eingesetzt.
Dort betreut sie des öfteren größere Fallkomplexe, für deren Abschluss Großdurchsuchungen nötig sind.
Die Momentanen Abläufe für das Vorbereiten und Organisieren wurden vor einiger Zeit von Frau Christensen entwickelt, trotzdem ist sie an einer Digitalisierung des Meldekopfs interessiert.

\subsubsection{Fokussierte Tätigkeit}

Bearbeitet Frau Christensen einen Fallkomplex welcher eine Großdurchsuchung benötigt, so ist sie verantwortlich diese vorzubereiten, jedoch nicht den Meldekopf dieser Durchsuchung zu übernehmen.
Nach vollendeter Durchsuchung werden ihr ein Protokoll des Meldekopfes, sowie die sichergestellten Unterlagen übermittelt, welche sie daraufhin auswertet.
Im Zuge des benannten Kontextes nimmt Frau Christensen also die Aufgabe des Vorbereitens der Durchsuchung war.
Die Organisation im Meldekopf wird von immer wechselnden Kollegen durchgeführt.
Auch Frau Christensen saß bereits bei Durchsuchungen anderer KollegInnen im Meldekopf.