\section{SOLL-Szenario}\label{sec:sollSzenario}

Aus den aus dem IST-Zustand gewonnen Informationen über die Tätigkeiten des Meldekopfes wird nun mit Hilfe der nutzungsorientierten Aspekte ein beispielhaftes SOLL-Szenario erstellt.
Dieses basiert auf den Abläufen des IST-Szenarios in \autoref{sec:istSzenario}.

Im realen Prozess der nutzungsorientierten Gestaltung werden viele Szenarien erstellt, welche den vollen Kontext abdecken.
Dies ist nötig, damit im Laufe der Entwicklung alle Arbeitsschritte und Tätigkeiten bedacht werden können.
Aus Gründen des Umfangs dieser Arbeit wird jedoch auf weitere Szenarien verzichtet.

Das folgende Szenario "Mehr Personal wird angefragt" beinhaltet die Persona Paul Theiss, welche in diesem Moment im Meldekopf tätig ist.
Genauer geht es um den Prozess einer Anforderung von mehr Personal zu einem Objekt.
In diesem SOLL-Szenario steht ihm die nutzungsorientierte Software zur Verfügung.

\subsection{SOLL-Szenario: Mehr Personal wird angefragt}

Paul ist momentan als Meldekopf für eine Durchsuchung eingeteilt. 
Er sitzt zusammen mit zwei Kollegen, Frau Beister und Herrn Diebold, in einem Besprechungsraum an einem großen tafelähnlichen Tisch. 
Im Raum steht ein Smart-Board, auf welchem der Übersichtsscreen der DCC geöffnet ist. 
Dieser enthält alle wichtigen Informationen zu allen an der Durchsuchung beteiligten Objekten. 
Die Ansicht kann auf Wunsch nach vielen Kriterien gefiltert werden. 
Momentan sitz Paul an seinem Platz und wartet auf eine Aufgabe. 
Vor sich hat Paul einen Laptop mit der Eingabemaske des DCC.

Pauls Telefon klingelt. 
Auf der anderen Seite ist Frau Schnee. 
Auf dem DCC sieht Paul, dass Frau Schnee momentan an Objekt D als Objektleiterin zugange ist. 
Frau Schnee erklärt Paul, dass ihr Team einen vorher nicht bekannten Keller unter dem Objekt gefunden hat, welcher ebenfalls durchsucht werden muss. 
Dazu möchte sie mehr Personal anfordern. 
Paul sieht direkt am DCC, dass momentan kein freies Personal verfügbar ist. 
Er entgegnet Frau Schnee, dass er sich melden wird, sobald er freies Personal für ihr Objekt findet. 

Über die Eingabemaske gibt Paul an, dass Objekt D Verstärkung benötigt. 
Dabei referenziert er das Objekt direkt und stellt die Dringlichkeit auf „wichtiges ToDo“. 
Ebenso notiert er den Fund des Kellers unter dem Objekt über die Eingabemaske, damit dies im Endprotokoll erscheint.

[Es vergeht etwas Zeit]

Wieder klingelt Pauls Telefon. 
Dieses Mal ist Herr Sebert, Objektleiter von Objekt B auf der anderen Seite. 
Dieser informiert Paul, dass die Maßnahmen an seinem Objekt abgeschlossen sind. 
Paul sieht auf dem Übersichtsscreen, dass Frau Schnee noch immer Personal an Objekt D benötigt. 
Mit einem Klick auf dieses sieht Paul die Adresse des Objekts, welche er Herrn Sebert mitteilt, damit er sich mit seinem Team auf den Weg machen kann. 
Dieser verabschiedet sich und macht sich mit seinen Kollegen auf den Weg zu Frau Schnee.

Paul nutzt nun den vorbereiteten Vorgang „verschieben“ in der Eingabemaske um das Team von Herrn Sebert zu Objekt D zu verschieben. 
Ebenso gibt er an, dass die Durchsuchung von Objekt B abgeschlossen ist. 
Dieses verschwindet aus der Übersicht. Zuletzt ruft Paul erneut bei Frau Schnee an, um diese über die kommenden Kollegen zu informieren. 
Frau Schnee bedankt sich und geht daraufhin zurück an ihre Arbeit.

Wenig später ruft erneut Herr Sebert an. 
Er teilt Paul mit, dass sein Team am neuen Einsatzort angekommen ist und sich ab jetzt in das Team von Frau Schnee eingliedert. 
Paul wählt in der Eingabemaske unter „Aktionen abschließen“ das Eintreffen des Teams an Objekt D aus, um die Ankunft auch für das Protokoll zu dokumentieren.
Damit ist für Paul die Aufgabe erledigt. 
