\section{IST-Szenario}\label{sec:istSzenario}

Aus den im Interview im Kontext, sowie dem Artefaktmodell gewonnen Informationen über die Tätigkeiten des Meldekopfes wird nun ein beispielhaftes Szenario erstellt.
Im realen Prozess der nutzungsorientierten Gestaltung werden viele Szenarien erstellt, welche den vollen Kontext abdecken.
Dies ist nötig, damit im Laufe der Entwicklung alle Arbeitsschritte und Tätigkeiten bedacht werden können.
Aus Gründen des Umfangs dieser Arbeit wird jedoch auf weitere Szenarien verzichtet.

Das folgende Szenario "Mehr Personal wird angefragt" beinhaltet die Persona Paul Theiss, welche in diesem Moment im Meldekopf tätig ist.
Genauer geht es um den Prozess einer Anforderung von mehr Personal zu einem Objekt.

\subsection{IST-Szenario: Mehr Personal wird angefragt}

Paul ist momentan als Meldekopf für eine Durchsuchung eingeteilt. 
Er sitzt zusammen mit zwei Kollegen, Frau Beister und Herrn Diebold, in einem Besprechungsraum an einem großen tafelähnlichen Tisch. 
Jedes von der Durchsuchung betroffenes Objekt wird durch ein DIN A4 Blatt auf dem Tisch vor ihm repräsentiert. 
Auf diesen kann Paul ein Bild des Objekts, alle diesem momentan zugeordneten Kollegen und einige Bemerkungen erkennen. 
Die Informationen sind sehr klein gedruckt. 
Paul muss also um den Tisch laufen, um alle Objekte betrachten zu können. 
An seinem Sitzplatz steht ein Telefon, sowie Stifte und Haftnotizen. 
Momentan sitzt Paul an seinem Platz und wartet auf eine Aufgabe.

Pauls Telefon klingelt. 
Auf der anderen Seite ist Frau Schnee.
Paul erinnert sich, dass Frau Schnee momentan an Objekt D zugange ist, also steht er auf und stellt sich vor die Informationen zu diesem Objekt. 
Aus dem Augenwinkel sieht er, dass Frau Schnee Objektleiterin von Objekt D ist. 
Diese erklärt Paul indessen, dass ihr Team einen vorher nicht bekannten Keller unter dem Objekt gefunden hat, welcher ebenfalls durchsucht werden muss. 
Dazu möchte sie mehr Personal anfordern. Paul entgegnet, dass er sich bemüht Personal zu finden und zurückrufen wird, sobald es Neues gibt.

Sobald Paul aufgelegt hat, fragt er zunächst seine beiden Kollegen nach momentan freien Kollegen, welche er zu Frau Schnee schicken kann. 
Diese verneinen die Nachfrage jedoch. Paul überfliegt nun alle Objektzettel. 
Genauer schaut er, ob unter „Bemerkungen“ vermerkt wurde, dass an einem Objekt entbehrliches Personal abgezogen werden kann. 
Leider ist dies nirgends der Fall. 
Paul nimmt sich also einen Stift und vermerkt sein neues Wissen zu Objekt D in dessen Bemerkungen. 
Ebenso merkt er dort an, dass mehr Personal nötig ist. 
Daraufhin ruft er Frau Schnee zurück, um ihr die schlechten Neuigkeiten mitzuteilen. 
Er will sie jedoch informieren, sobald Kollegen frei werden.

[Es vergeht etwas Zeit]

Wieder klingelt Pauls Telefon. 
Dieses Mal ist Herr Sebert, Objektleiter von Objekt B auf der anderen Seite. 
Dieser informiert Paul, dass die Maßnahmen an seinem Objekt abgeschlossen sind. 
Paul bedankt sich für die Information und möchte Herrn Siebert samt seinen Kollegen schon zurück in die Dienststelle schicken als ihm einfällt, dass Frau Schnee noch immer auf Unterstützung wartet. 
Dies teilt Paul Herrn Sebert mit. 
Er läuft zu den Informationen zu Objekt D, um Herrn Sebert die Adresse des neuen Objekts mitzuteilen. 
Dieser verabschiedet sich und macht sich mit seinen Kollegen auf den Weg zu Frau Schnee.
Paul macht einen großen Hacken an den Zettel zu Objekt B um dieses für alle im Raum als abgeschlossen zu kennzeichnen. 
Daraufhin nimmt der die Papierstreifen der diesem Objekt zugeordneten Kollegen und legt diese zu Objekt D. 
In dessen Bemerkungen notiert er die Verschiebung der Kollegen samt aktueller Zeit. 
Zuletzt ruft er erneut bei Frau Schnee an, um diese über die kommenden Kollegen zu informieren. 
Frau Schnee bedankt sich und geht daraufhin zurück an ihre Arbeit.

Wenig später ruft erneut Herr Sebert an. 
Er teilt Paul mit, dass sein Team am neuen Einsatzort angekommen ist und sich ab jetzt in das Team von Frau Schnee eingliedert. 
Damit ist für Paul die Aufgabe erledigt. 
Dies kennzeichnet er mit einem Hacken hinter dem vermerkten Wunsch auf Unterstützung in den Bemerkungen von Objekt D.
